\chapter{Système d'acquisition de données}
\label{chap:acquisition}

Le système d'acquisition a pour objectif de faire la sauvegarde et la
transmission au sol de données obtenues à l'aide de capteurs à bord de la fusée.
Les données prises sont:

\begin{itemize}
	\item l'altitude
	\item la température
	\item les coordonnées GPS (latitude, longitude)
	\item l'accélération sur les axes x, y et z
	\item le champ magnétique sur les axes x, y et z
	\item la vitesse angulaire sur les axes de rotation x, y et z
\end{itemize}

% TODO: inclure photo du board

\section{Réalisation}

Cette section présente la réalisation du système d'acquisition, selon les
capteurs utilisés, le microcontrôleur, le système de sauvegarde des données et
le système de transmission au sol.

\subsection{Support physique}

Depuis 2017, nous sommes obligés d'insérer l'avionique dans un support qui
respecte le format CubeSat. De plus, cette année il a été décidé que le système
d'acquisition de données reposerait sur un seul circuit imprimé. Ce circuit
devait comprendre toutes les composantes du système et être de forme
rectangulaire, ayant moins de 10~cm de longueur de côté. Pour la connection
entre les circuits, nous avons changé les connections de type PCIe de l'an
dernier pour utiliser des câbles plats. Un avantage marquant est la liberté
offerte au niveau du placement des circuits dans le CubeSat. En effet, comme
la longueur des câbles est ajustable et que les câbles sont flexibles, le
placement des circuits est beaucoup permissif que l'an dernier, où il fallait
avoir une grande précision sur le placement des connecteurs.

\subsection{Capteurs}



\subsection{Microcontrôleur}



\subsection{Sauvegarde des données}



\subsection{Transmission au sol}



\section{Rétrospective}

\subsection{Points forts}

L'utilisation d'un microcontrôleur offrant une interface de debug s'est avérée
très utile durant les quelques semaines avant la compétition. En effet, cette
fonctionnalité diminue grandement le temps de développement et justifie
amplement la courbe d'apprentissage lors du passage d'Arduino à un autre
microcontrôleur.
\par


\subsection{Points à améliorer}

La méthode d'acquisition des données GPS pourrait être améliorée. Au lieu
d'utiliser une méthode d'écoute passive sur la ligne de transmission du GPS, on
pourrait considérer d'y aller par \textit{polling}, i.e. demander au GPS
lorsqu'on a besoin des données. Cela permettrait d'économiser de l'énergie,
puisque le GPS ne transmettrait pas en continue, en plus de garder un meilleur
contrôle sur ses données dans le programme.
\\
\par
Le code pour la sauvegarde des données ne fonctionne pas. Le choix des broches
pour la communication SPI avec la carte SD est peut-être en cause, comme montré
par le logiciel CubeMX. Plus de tests sont à faire, puisque la cause du
problème est totalement inconnue.
