\chapter{Système d'acquisition de données}
\label{chap:acquisition}

Le système d'acquisition a pour objectif de faire la sauvegarde et la
transmission au sol de données obtenues à l'aide de capteurs à bord de la fusée.
Les données prises sont:

\begin{itemize}
	\item l'altitude
	\item la température
	\item les coordonnées GPS (latitude, longitude)
	\item l'accélération sur les axes x, y et z
	\item le champ magnétique sur les axes x, y et z
	\item la vitesse angulaire sur les axes de rotation x, y et z
\end{itemize}

% TODO: inclure photo du board

\section{Réalisation}

Cette section présente la réalisation du système d'acquisition, selon les
capteurs utilisés, le microcontrôleur, le système de sauvegarde des données et
le système de transmission au sol.

\subsection{Capteurs}



\subsection{Microcontrôleur}



\subsection{Sauvegarde des données}



\subsection{Transmission au sol}



\section{Rétrospective}

\subsection{Points forts}

L'utilisation d'un microcontrôleur offrant une interface de debug s'est avérée
très utile durant les quelques semaines avant la compétition. En effet, cette
fonctionnalité diminue grandement le temps de développement et justifie
amplement la courbe d'apprentissage lors du passage d'Arduino à un autre
microcontrôleur.
\par



\subsection{Points à améliorer}
