\chapter{Antenne patch et Yagi}
\label{chap:antennes}

\section{Réalisation}

Pour la transmission au sol des données de vol, 2 antennes ont été conçues: une
antenne patch et une antenne Yagi-Uda. La réalisation de chacune de ces
antennes sera présentée dans les prochaines sections.

\subsection{Antenne patch}

L'antenne patch est essentiellement une feuille de cuivre, découpée en une
forme prédéterminée et fixée sur un diélectrique. Le diélectrique utilisé est
le teflon. Tel que montré dans la figure \ref{f:antenne_patch}, la feuille de
cuivre a été découpée en forme rectangulaire, avec des dimensions choisies de
façon à transmettre dans la bande des 900 MHz.

\begin{figure}[H]
	\center
	\includegraphics[width=0.7\textwidth]{fig/avionique/patch_antenna.png}
	\caption{L'antenne patch}
	\label{f:antenne_patch}
\end{figure}

\subsection{Antenne Yagi-Uda}

L'antenne Yagi-Uda est composé d'un support de bois, sur lequel plusieurs tiges
de métal sont alignées parallèlement. La 2ème de ces tiges en partant du bas
est le récepteur, tandis que les autres tiges guident le signal. À noter que
cette antenne est directionnelle et seul le récepteur est connecté au RFD900.
La figure \ref{f:diagramme_directivite_yagi} montre la directivité de l'antenne
Yagi-Uda du Gaul. La directivité est d'environ 6~dB. Comme on le voit, le lobe
de devant est très bien défini. Le lobe de dernière l'est moins, mais l'effet
est négligeable puisque ce lobe est moins important.

\begin{figure}[H]
	\center
	\includegraphics[width=0.7\textwidth]{fig/avionique/diagramme-directivite-Yagi.png}
	\caption{Diagramme de directivité de l'antenne Yagi-Uda}
	\label{f:diagramme_directivite_yagi}
\end{figure}

\section{Rétrospective}

\subsection{Points forts}

Les 2 antennes ont très bien fonctionné durant les tests d'intégration. Il
était possible de transmettre des données du prototype du système d'acquisition
vers la station au sol sans problèmes.

\subsection{Points à améliorer}

Comme le système a fonctionné parfaitement dans le contexte des tests auquel il
a été soumis, les recommendations suivantes sont plus des améliorations pour
l'an prochain que des corrections à apporter.
\par
La première amélioration possible est de fabriquer l'antenne patch sur un
circuit imprimé flexible. Bien que ce soit une option esthétique très
intéressante, il faut s'assurer de l'exactitude des dimensions et de la
constante diélectrique, de l'épaisseur du substrat, etc, avant de procéder. Des
informations ont déjà été récoltées à ce sujet et Circuits Imprimés de la
Capitale serait disposé à nous commanditer cette impression.
\par
À l'aide de notre commandite Matlab qui nous permet d'utiliser l'Antenna
Toolbox, il serait intéressant d'étudier d'autres possibilités de conception
d'antennes. Rappel : Les modems RFD900+ utilisés au GAUL fonctionnent dans la
bande de fréquence 902-928 MHz, ce qui donne une fréquence centrale de 915 MHz.
Le design doit être réalisé en conséquence. La polarisation également est
importante, elle est de type linéaire pour ces modules de communication.
