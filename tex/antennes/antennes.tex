\chapter{Antenne patch et Yagi}
\label{chap:antennes}

\section{Réalisation}


\begin{figure}[H]
	\center
	\includegraphics[width=0.7\textwidth]{fig/diagramme-directivite-Yagi.png}
	\caption{Diagramme de directivité de l'antenne Yagi-Uda}
	\label{f:diagramme_directivite_yagi}
\end{figure}

\section{Rétrospective}

\subsection{Points forts}

Les 2 antennes ont très bien fonctionné durant les tests d'intégration. Il
était possible de transmettre des données du prototype du système d'acquisition
vers la station au sol sans problèmes.

\subsection{Points à améliorer}

Comme le système a fonctionné parfaitement dans le contexte des tests auquel il
a été soumis, les recommendations suivantes sont plus des améliorations pour
l'an prochain que des corrections à apporter.
\par
La première amélioration possible est de fabriquer l'antenne patch sur un
circuit imprimé flexible. Bien que ce soit une option esthétique très
intéressante, il faut s'assurer de l'exactitude des dimensions et de la
constante diélectrique, de l'épaisseur du substrat, etc, avant de procéder. Des
informations ont déjà été récoltées à ce sujet et Circuits Imprimés de la
Capitale serait disposé à nous commanditer cette impression.
\par
À l'aide de notre commandite Matlab qui nous permet d'utiliser l'Antenna
Toolbox, il serait intéressant d'étudier d'autres possibilités de conception
d'antennes. Rappel : Les modems RFD900+ utilisés au GAUL fonctionnent dans la
bande de fréquence 902-928 MHz, ce qui donne une fréquence centrale de 915 MHz.
Le design doit être réalisé en conséquence. La polarisation également est
importante, elle est de type linéaire pour ces modules de communication. 
