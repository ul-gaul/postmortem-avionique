\chapter{Système de déploiement des parachutes}
\label{chap:deploiement}

Le système de déploiement des parachutes a pour objectif d'offrir un moyen
fiable de déployer les parachutes de la fusée au bon moment, en plus d'être
sécuritaire et simple. Le système de déploiement est composé de 2 circuits: un
circuit développé par le GAUL (déploiement maison) et un circuit agissant comme
redondance, un \textit{Statologger SL100} de \textit{PerfectFlite} (déploiement
commercial).

\section{Réalisation}

\subsection{Déploiement maison}


Tel que le montre la figure \ref{f:diag_fonc_deploiement_maison}, le système de
déploiement maison est composé de 6 sous-systèmes: l'alimentation, le
microcontrôleur, l'altimètre, la sortie sonore (buzzer), les allumettes
électroniques et le point de sauvegarde des données de vol. Tous ces
sous-systèmes, exception faite de l'alimentation qui est couverte au chapitre
\ref{chap:alimentation}, seront détaillés dans les sections suivantes.

\begin{figure}[H]
	\center
	\includegraphics[width=0.8\textwidth]{fig/diag_fonc_deploiement_maison.png}
	\caption{Diagramme fonctionnel du système de déploiement maison}
	\label{f:diag_fonc_deploiement_maison}
\end{figure}

\subsubsection{Microcontrôleur}



\subsubsection{Altimètre}

Afin de mesurer l'altitude en temps réel, un altimètre barométrique est
utilisé, plus précisément, le breakout board BMP180 de Sparkfun illustré à la
figure \ref{f:BMP180}.

\begin{figure}[H]
	\center
	\includegraphics[totalheight=0.20\textheight]{fig/BMP180.jpg}
	\caption{Breakout board du BMP180 de Sparkfun}
	\label{f:BMP180}
\end{figure}

La librairie "Adafruit\_BMP085.h" est utilisée afin de simplifier la
manipulation du capteur par le microcontrôleur.

\subsubsection{Buzzer}



\subsubsection{Allumettes électroniques}

Pour le déploiement des parachutes, une étincelle est nécessaire pour faire
sauter une petite quantité de poudre noire. Afin de produire cette étincelle,
une allumette électronique comme celle illustrée à la figure \ref{f:e-match} est
utilisée. Pour la faire sauter, il suffit de lui envoyer un courant supérieur
à une valeur typique fourni par le fabriquant, normalement de l'ordre des
100 mA.

\begin{figure}[H]
	\center
	\includegraphics[width=0.8\textwidth]{fig/e-match.jpg}
	\caption{Allumette électronique}
	\label{f:e-match}
\end{figure}

\subsubsection{Sauvegarde des données de vol}



\subsection{Déploiement commercial}



\section{Rétrospective}

\subsection{Points forts}



\subsection{Points à améliorer}
