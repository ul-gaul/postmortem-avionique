\chapter{Système de déploiement des parachutes}
\label{chap:deploiement}

Le système de déploiement des parachutes a pour objectif d'offrir un moyen
fiable de déployer les parachutes de la fusée au bon moment, en plus d'être
sécuritaire et simple. Le système de déploiement est composé de 3 circuits: un
circuit développé par le GAUL (déploiement maison), un circuit agissant comme
redondance, un \textit{Statologger SL100} de \textit{PerfectFlite} (déploiement
commercial) et finalement le circuit de la manette contrôlant les relais
d'alimentation.

\section{Réalisation}

\subsection{Déploiement maison}


Tel que le montre la figure \ref{f:diag_fonc_deploiement_maison}, le système de
déploiement maison est composé de 6 sous-systèmes: l'alimentation, le
microcontrôleur, l'altimètre, la sortie sonore (buzzer), les allumettes
électroniques et le point de sauvegarde des données de vol. Tous ces
sous-systèmes seront détaillés dans les sections suivantes.

\begin{figure}[H]
	\center
	\includegraphics[width=0.8\textwidth]{fig/avionique/diag_fonc_deploiement_maison.png}
	\caption{Diagramme fonctionnel du système de déploiement maison}
	\label{f:diag_fonc_deploiement_maison}
\end{figure}

\subsubsection{Alimentation}

Le circuit de déploiement est alimenté par une batterie 9~volts. Un relai
électromagnétique contrôle l'alimentation. Ce relai est connecté au circuit
d'acquisition de données par un câble plat et c'est le circuit d'acquisition
qui contrôle l'état du relai. La batterie est maintenue en place par son
support, renforcé par des attaches de type \textit{tie wrap}.
\\
\par
Lorsque l'alimentation est activée, tous les systèmes sont alimentés et le
microcontrôleur entre dans sa boucle de contrôle.

\subsubsection{Microcontrôleur}

Le microcontrôleur utilisé est un ATmega328p. Le code qui s'exécute sur celui-ci
est organisé selon l'architecture montrée à la figure \ref{f:arch_deploiement}.

\begin{figure}[H]
	\center
	\includegraphics[width=0.8\textwidth]{fig/avionique/architecture_deploiement.png}
	\caption{Architecture du programme de déploiement}
	\label{f:arch_deploiement}
\end{figure}

Lors d'un vol, les étapes suivantes s'enchaînent selon des événements déterminés
par l'altitude filtrée et la vitesse de la fusée:

\begin{itemize}
	\item Launchpad: avant le décollage
	\item Burnout: durant la montée propulsée par le moteur
	\item Pre\_drogue: après la propulsion du moteur, mais avant l'apogée
	\item Pre\_main: après l'apogée, avant le déploiement du main
	\item Drift: après le déploiement du main, avant de toucher le sol
	\item Landed: la fusée a atterri
\end{itemize}


L'implémentation du code et de la machine à état du vol est disponible
\href{https://github.com/ul-gaul/deploiement}{ici} pour consultation.

\subsubsection{Altimètre}

Afin de mesurer l'altitude en temps réel, un altimètre barométrique est
utilisé, plus précisément, le breakout board BMP180 de Sparkfun illustré à la
figure \ref{f:BMP180}.

\begin{figure}[H]
	\center
	\includegraphics[totalheight=0.20\textheight]{fig/avionique/BMP180.jpg}
	\caption{Breakout board du BMP180 de Sparkfun}
	\label{f:BMP180}
\end{figure}

La librairie "Adafruit\_BMP085.h" est utilisée afin de simplifier la
manipulation du capteur par le microcontrôleur.

\subsubsection{Buzzer}

Le buzzer du circuit de déploiement maison est utilisé pour communiquer l'état
du vol par moyen audio. L'état est représenté par un nombre de courts bips
répétés à intervalles réguliers. Les états sont:

\begin{enumerate}
	\item Tous les parachutes sont connectés
	\item Le drogue est sorti
	\item Le main est sorti
	\item Le vol est terminé
\end{enumerate}


\subsubsection{Allumettes électroniques}

Pour le déploiement des parachutes, une étincelle est nécessaire pour faire
sauter une petite quantité de poudre noire. Afin de produire cette étincelle,
une allumette électronique comme celle illustrée à la figure \ref{f:e-match} est
utilisée. Pour la faire sauter, il suffit de lui envoyer un courant supérieur
à une valeur typique fourni par le fabriquant, normalement de l'ordre des
100 mA.

\begin{figure}[H]
	\center
	\includegraphics[width=0.8\textwidth]{fig/avionique/e-match.jpg}
	\caption{Allumette électronique}
	\label{f:e-match}
\end{figure}

\subsubsection{Sauvegarde des données de vol}

La sauvegarde des données de vol se fait sur une carte micro-SD. Le
microcontrôleur utilise la librairie \textit{SD.h} fournie par Arduino, afin
de gèrer le système de fichier FAT32. Les données sauvegardées sont l'altitude
brute (non-filtrée), l'altitude filtrée, l'altitude maximale atteinte et la
vitesse.

\subsection{Déploiement commercial}

Le déploiement commercial est essentiellement un \textit{Statologger SL100}
de \textit{PerfectFlite}, tel que montré à la figure \ref{f:stratologger},
alimenté par une batterie 9~volts. La commutation de l'alimentation est
effectuée par le même système de relais que pour le déploiement maison.

\begin{figure}[H]
	\center
	\includegraphics[width=0.8\textwidth]{fig/avionique/stratologger.png}
	\caption{Stratologger SL100 de PerfectFlite}
	\label{f:stratologger}
\end{figure}

Lorsque le vol est terminé, le stratologger utilise son buzzer pour communiquer
l'altitude de l'apogée. Lors de la compétition de 2018, il a été observé que
ce système a une autonomie d'au moins 24~heures.

\subsection{Manette de contrôle des relais}

% TODO: photo de la manette quelque part

La manette de contrôle des relais est un système simple de communication UART,
fonctionnant selon un modèle \textit{master-slave}, où la manette est le
\textit{master} (maître) et le système d'acquisition agit comme \textit{slave}
(esclave).
\\
\par
Quant au protocol utilisé pour la communication des commandes, celui-ci
consiste en l'envoi d'un octet par la manette et le retour d'un octet par le
système d'acquisition. La table \ref{t:protocol_manette} montre les commandes
supportées par la manette.

\begin{table}[H]
	\centering
	\begin{tabular}{|c|c|p{4cm}|c|}
		\hline
		Commande & Valeur & Description & Temps d'exécution \\ \hline
		\texttt{GET RELAY STATE} & \texttt{0xAX - 1010 XXXX} & \parbox{4cm}{
		Demande l'état (set ou reset) du relai X.} & de 1.1~ms à 100~ms \\ \hline
		\texttt{SET RELAY} & \texttt{0xBX - 1011 XXXX} & \parbox{4cm}{Met le
		relai X à l'état set, état où il alimente son circuit.} & de 1~ms à
		100~ms \\ \hline
		\texttt{RESET RELAY} & \texttt{0xCX - 1100 XXXX} & \parbox{4cm}{Met le
		relai X à l'état reset, état où il n'alimente plus son circuit.} & de
		1~ms à 100~ms \\ \hline
		\texttt{RESERVED} & \texttt{0xYX - YYYY XXXX} & Commandes réservées &
		\textbf{N/A} \\ \hline
	\end{tabular}
	\caption{Commandes supportées par le protocol de contrôle des relais
	d'alimentation}
	\label{t:protocol_manette}
\end{table}

Le code de cette implémentation est disponible
\href{https://github.com/ul-gaul/manette}{ici} pour consultation. Il contient
le code de la manette et le code \textit{slave} écrit pour un Arduino Uno.

\section{Rétrospective}

\subsection{Points forts}

\subsubsection{Interconnections entre les circuits}
L'utilisation de câble plat pour interconnecter le circuit d'acquisition avec
les circuits de déploiement était une très bonne idée. Elle permet d'ajuster la
hauteur des circuits dans le cubesat, de conserver la conception modulaire du
système d'avionique, en plus d'être facile à comprendre et utiliser.

\subsection{Points à améliorer}

\subsubsection{Déploiement maison}

L'outil de simulation Python devrait pouvoir générer le code C de simulation
pour accélérer le développement est les tests. La simulation elle-même pourrait
utiliser des données de plusieurs autres vols et faire de meilleurs calculs
d'interpolation.
\\
\par
Pour le circuit maison, comme il n'a pas pu être testé cette année, les points
suivants sont basés sur le déploiement de l'an dernier. Pour améliorer le mode
de simulation du circuit, il serait pertinent d'enrichir les données
sauvegardées sur la carte SD. Aussi, des LEDs d'état ou un écran LCD seraient
un bon moyen de suivre la simulation sans avoir à brancher des LEDs directement
dans les prises destinées aux allumettes électroniques.

\subsubsection{Manette des relais}

L'affichage de la manette pourrait être améliorer en remplaçant les LEDs d'état
par un écran LCD. De plus, les boutons devraient être changés, car ils n'ont
pas une très bonne rétroaction lors de l'appui, ce qui est embêtant lors des
tests, encore plus lors de l'utilisation ultime de la manette. Finalement, la
disposition des boutons, piles et switches sur le circuit pourrait être revue
afin de rendre son utilisation plus facile.
