\chapter{Système de déploiement des parachutes}
\label{chap:deploiement}

Le système de déploiement des parachutes a pour objectif d'offrir un moyen
fiable de déployer les parachutes de la fusée au bon moment, en plus d'être
sécuritaire et simple. Le système de déploiement est composé de 3 circuits: un
circuit développé par le GAUL (déploiement maison), un circuit agissant comme
redondance, un \textit{Statologger SL100} de \textit{PerfectFlite} (déploiement
commercial) et finalement le circuit de la manette contrôlant les relais
d'alimentation.

\section{Réalisation}

\subsection{Déploiement maison}


Tel que le montre la figure \ref{f:diag_fonc_deploiement_maison}, le système de
déploiement maison est composé de 6 sous-systèmes: l'alimentation, le
microcontrôleur, l'altimètre, la sortie sonore (buzzer), les allumettes
électroniques et le point de sauvegarde des données de vol. Tous ces
sous-systèmes seront détaillés dans les sections suivantes.

\begin{figure}[H]
	\center
	\includegraphics[width=0.8\textwidth]{fig/diag_fonc_deploiement_maison.png}
	\caption{Diagramme fonctionnel du système de déploiement maison}
	\label{f:diag_fonc_deploiement_maison}
\end{figure}

\subsubsection{Alimentation}

Le circuit de déploiement est alimenté par une batterie 9~volts. Un relai
électromagnétique contrôle l'alimentation. Ce relai est connecté au circuit
d'acquisition de données par un câble plat et c'est le circuit d'acquisition
qui contrôle l'état du relai. La batterie est maintenue en place par son
support, renforcé par des attaches de type \textit{tie wrap}.
\\
\par
Lorsque l'alimentation est activée, tous les systèmes sont alimentés et le
microcontrôleur entre dans sa boucle de contrôle.

\subsubsection{Microcontrôleur}



\subsubsection{Altimètre}

Afin de mesurer l'altitude en temps réel, un altimètre barométrique est
utilisé, plus précisément, le breakout board BMP180 de Sparkfun illustré à la
figure \ref{f:BMP180}.

\begin{figure}[H]
	\center
	\includegraphics[totalheight=0.20\textheight]{fig/BMP180.jpg}
	\caption{Breakout board du BMP180 de Sparkfun}
	\label{f:BMP180}
\end{figure}

La librairie "Adafruit\_BMP085.h" est utilisée afin de simplifier la
manipulation du capteur par le microcontrôleur.

\subsubsection{Buzzer}



\subsubsection{Allumettes électroniques}

Pour le déploiement des parachutes, une étincelle est nécessaire pour faire
sauter une petite quantité de poudre noire. Afin de produire cette étincelle,
une allumette électronique comme celle illustrée à la figure \ref{f:e-match} est
utilisée. Pour la faire sauter, il suffit de lui envoyer un courant supérieur
à une valeur typique fourni par le fabriquant, normalement de l'ordre des
100 mA.

\begin{figure}[H]
	\center
	\includegraphics[width=0.8\textwidth]{fig/e-match.jpg}
	\caption{Allumette électronique}
	\label{f:e-match}
\end{figure}

\subsubsection{Sauvegarde des données de vol}



\subsection{Déploiement commercial}

Le déploiement commercial est essentiellement un \textit{Statologger SL100}
de \textit{PerfectFlite}, tel que montré à la figure \ref{f:stratologger},
alimenté par une batterie 9~volts. La commutation de l'alimentation est
effectuée par le même système de relais que pour le déploiement maison.

\begin{figure}[H]
	\center
	\includegraphics[width=0.8\textwidth]{fig/stratologger.png}
	\caption{Stratologger SL100 de PerfectFlite}
	\label{f:stratologger}
\end{figure}

Lorsque le vol est terminé, le stratologger utilise son buzzer pour communiquer
l'altitude de l'apogée. Lors de la compétition de 2018, il a été observé que
ce système a une autonomie d'au moins 24~heures. 

\subsection{Manette de contrôle des relais}



\section{Rétrospective}

\subsection{Points forts}



\subsection{Points à améliorer}
