\chapter{Station au sol - Logiciel}
\label{chap:sas}

\section{Réalisation}

Le logiciel de la station au sol permet de recevoir les données de vol transmises par la fusée, d'afficher ces données en temps réel dans un interface graphique riche, de les sauvegarder dans un fichier et, enfin, de rejouer un vol.
Deux modes d'utilisation sont offerts à l'utilisateur: le mode temps réel et le mode replay.
Le code source de ce programme est écrit en \textit{Python 3}.
Celui-ci est portable et donc peut s'exécuter sur un
ordinateur portable autant que sur le Raspberry Pi de la station au sol
matérielle.
Les sections suivantes détaillent le fonctionnement de ce logiciel.

\subsection{Architecture producteur-consommateur}
La gestion du flot de données est basée sur une architecture producteur-consommateur.
Le producteur est responsable de produire les données de vol de façon continue et régulière, sous forme de paquets appelés \emph{RocketPackets}.
Deux types de producteurs existent, soient le producteur temps réel et le producteur replay.
Chacun d'eux est une implémentation d'une classe producteur abstraite.
Ainsi, les deux types de producteur sont interchangeables du point de vue du consommateur et ils sont libres de générer les nouveaux \emph{RocketPackets} à leur façon.
Le producteur temps réel lit le port série sur lequel est connectée l'antenne de réception et produit les paquets à partir du flot de données reçues en temps réel de la fusée (voir la section \ref{s:rocket_communication}).
Le producteur replay génère les paquets en lisant le fichier de sauvegarde d'un vol et simule une réception temps réel (voir la section \ref{s:mode_replay}).

Le consommateur, quand à lui, est unique et ne dépend pas du mode d'utilisation du logiciel.
Il consomme les paquets et extrait chacun des champs du \emph{RocketPackets}, qu'il groupe en listes.
Le consommateur effectue également les différents traitements sur ces données pour les rendre utilisables par les différents widgets de l'interface graphique (voir section \ref{s:traitement}).

Finalement, un contrôleur se charge d'orchestrer le producteur et le consommateur, en plus de rafraîchir l'interface graphique à intervalles réguliers et de gérer les entrées de l'utilisateur.

\subsection{Communication avec la fusée}
\label{s:rocket_communication}
\textbf{TODO: updater le texte pour le forat 2018 du RocketPacket}

La communication entre la fusée et l'ordinateur de la station au sol s'effectue à l'aide d'un émetteur-récepteur de type RFD-900.
Ce module de communication transfert les données reçues à l'ordinateur portable ou au Raspberry Pi par son port USB.
Cette communication est possible grâce à un protocole d'échange de \emph{RocketPackets}, qui sont des messages de longueur fixe dont le format est défini dans le tableau \ref{LogicielRealisationFormatPacket}.
Chaque \emph{RocketPacket} débute par un octet de démarrage, soit le caractère 's' en ASCII, et se termine par un octet de validation.
\\ \\
L'algorithme de réception des données est l'exécution en boucle des étapes suivantes:

\begin{enumerate}
    \item Lire un à un les octets du buffer du port USB jusqu'à obtenir l'octet de démarrage
    \item Lire les 109 octets du \emph{RocketPacket}
    \item Calculer la somme de contrôle
    \item Si le résultat de la somme de contrôle est valide, désérialiser l'objet \emph{RocketPacket} et ajouter celui-ci à la liste des paquets reçus.
    \item Sinon, rejeter le paquet.
\end{enumerate}

L'algorithme employé pour assurer la validité des données est une somme de contrôle très simple.
L'envoyeur additionne les 108 octets des données du paquet (excluant l'octet de démarrage), puis effectue une inversion logique du résultat.
L'octet obtenu est placé à la fin du \emph{RocketPacket}.
Lorsque le receveur reçoit le paquet, il additionne ces 109 octets.
Comme l'octet de validation est le complément de la somme sur 8 bits des 108 premiers octets, le résultat attendu est 255 (1111 1111).
Si le receveur obtient ce résultat, il conclut que les données ne sont pas corrompues.

\begin{table}
    \centering
    \begin{tabular}{|l|l|}
        \hline
        \textbf{Champ} & \textbf{Nombre d'octets} \\
        \hline
        Octet de départ & 1 \\
        &\\
        Timestamp & 4 \\
        Vitesse angulaire (x) & 4 \\
        Vitesse angulaire (y) & 4 \\
        Vitesse angulaire (z) & 4 \\
        Accélération (x) & 4 \\
        Accélération (y) & 4 \\
        Accélération (z) & 4 \\
        Altitude & 4 \\
        Latitude 1 & 4 \\
        Longitude 1 & 4 \\
        Latitude 2 & 4 \\
        Longitude 2 & 4 \\
        Température & 4 \\
        Date (heures) & 4 \\
        Date (minutes) & 4 \\
        Date (secondes) & 4 \\
        Quaternion (a) & 4 \\
        Quaternion (b) & 4 \\
        Quaternion (c) & 4 \\
        Quaternion (d) & 4 \\
        État du board d'acquisition 1 & 1 \\
        État du board d'acquisition 2 & 1 \\
        État du board d'acquisition 3 & 1 \\
        État du board d'alimentation 1 & 1 \\
        État du board d'alimentation 2 & 1 \\
        État du board du payload & 1 \\
        Non utilisé & 4 \\
        Non utilisé & 4 \\
        Vitesse angulaire payload (x) & 4 \\
        Vitesse angulaire payload (y) & 4 \\
        Vitesse angulaire payload (z) & 4 \\
        Non utilisé & 1 \\
        Non utilisé & 1 \\
        &\\
        Octet de validation & 1 \\
        \hline
    \end{tabular}
    \caption{Format du RocketPacket}
    \label{LogicielRealisationFormatPacket}
\end{table}

\subsection{Sauvegarde et chargement de données de vol}
\label{s:sauvegarde}
Au cours du vol de la fusée, le logiciel acquiert des données et les affiche.
À la fin du vol, au moment d'arrêter l'acquisition, le logiciel offre la possibilité à l'utilisateur de sauvegarder les données dans un fichier CSV.
Le format de ce fichier est très simple.
Chaque ligne du fichier correspond à un \emph{RocketPacket} et chacun des champs d'un paquet sont écrits sur cette ligne, dans un ordre précis et séparés par des virgules.
Ainsi, dans une colonne se trouve le même champ pour tous les \emph{RocketPackets}.
La première ligne du fichier contient les entêtes, c'est-à-dire le nom sous forme de texte du champ présent dans cette colonne.

Lorsque l'utilisateur veut charger les données d'un vol, il spécifie le fichier CSV correspondant.
Le logiciel parcours alors les lignes du fichier, désérialise les champs et recrée les \emph{RocketPacket} à partir de ceux-ci.

\subsection{Mode replay}
\label{s:mode_replay}
Le logiciel offre la possibilité de rejouer un vol précédent en chargeant les données sauvegardées dans un fichier CSV tel que décrit à la section \ref{s:sauvegarde}.
Une fois le fichier lu, le producteur de type replay possède la totalité des \emph{RocketPackets} de ce vol.
Afin de simuler une acquisition temps réel, il rend les paquets disponibles au consommateur graduellement, en se basant sur leur timestamp.
Ce champ correspond au nombre de millisecondes écoulées depuis le début de l'acquisition.
Le producteur replay possède sont propre fil d'exécution.
Il commence par \emph{produire} un premier paquet, puis calcule le temps entre le timestamp de celui-ci et le timestamp du paquet suivant.
Le producteur s'endort pour cet intervalle de temps, puis se réveil et refait le même traitement pour le paquet suivant.
Il continue ainsi jusqu'au dernier \emph{RocketPacket}, puis se met en veille.

Le mode replay offre à l'utilisateur des contrôles similaires à ceux d'un lecteur vidéo.
En effet, il est possible de mettre la lecture du vol sur pause, puis de reprendre le visionnement.
De plus, le mode replay offre la possibilité de faire une avance rapide et même de reculer, comme avec un vidéo.
Le producteur replay possède un membre de type \emph{PlaybackState}, qui est une structure de données comportant le mode (avancer ou reculer) ainsi que la vitesse (1x, 2x, 4x ou 8x) de lecture.
Lorsque l'utilisateur demande de reculer, le producteur retire le dernier paquet de la liste des paquets disponibles au lieu d'en ajouter un autre et se base sur le timestamp du paquet précédent pour calculer son temps d'attente.
Enfin, pour changer la vitesse de lecture, le temps d'attente entre les paquets est divisé par le facteur de vitesse.
Par exemple, s'il y a une seconde d'écart entre les timestamps de deux \emph{RocketPackets} et que la vitesse de lecture est à 2x, alors le producteur ne se mettra en veille que pour 0.5 seconde avant de produire le paquet suivant.

\subsection{Traitement des données}
\label{s:traitement}
Après avoir obtenu les \emph{RocketPackets} du producteur, le consommateur a la responsabilité de traiter ces données afin de les rendre utilisables par l'interface graphique.
Tout d'abord, le consommateur obtient une liste de \emph{RocketPackets}, qu'il transforme en plusieurs listes, contenant chacune les valeurs d'un certain champ du paquet.
On a donc une liste contenant toutes les valeurs de timestamps, une autre pour les valeurs d'altitude, une autre pour la latitude, la longitude et ainsi de suite.

Par la suite, les valeurs d'altitude en mètres sont converties en pieds, car c'est l'unité de mesure couramment utilisée dans le cadre de la compétition.
De plus, les coordonnées GPS sous forme de latitudes et longitudes sont converties en coordonnées du systèmes de projection Transverse Universelle de Mercator (UTM).
Ce système de coordonnées est utilisé, car il s'agit d'une projection utilisant un plan cartésien en deux dimensions, contrairement au système traditionnel de latitude et longitude.
Cela permet donc de représenter plus facilement sur un graphique la position de la fusée par rapport à la rampe de lancement.

Enfin, le consommateur calcule la position de la rampe de lancement à partir  des données de position de la fusée.
Pour ce faire, il fait une moyenne de la position de la fusée lors des dix premières secondes d'acquisition.
On assume que l'acquisition est démarrée plusieurs minutes avant le décollage, alors que la fusée est sur la rampe.
Par la suite, cette position est considérée comme l'origine du graphique et la position de la fusée est exprimée par rapport à celle-ci.

\subsection{Interface graphique}


\section{Rétrospective}

\subsection{Points forts}


\subsection{Points à améliorer}
