\chapter{Station au sol - SAS}
\label{chap:sas}

\section{Réalisation}

La station au sol est composé principalement de 2 parties: la partie matérielle
et la partie logicielle. La réalisation de ces 2 parties sera détaillée dans
les sections suivantes.

\subsection{Station au sol - Matérielle}

Les sections suivantes détaillent chaque composante/système de la SAS
matérielle. Un liste complète des pièces est disponible sur le Google Drive du
Gaul, dans le document \href{https://drive.google.com/open?id=1WfV-Swc37Ih476rmRLYdwrBWRaUo8ZzXshS-LzYTQF4}{Rouleau sacré de la SAS}.

\subsubsection{Boîtier}

Le boîtier de la SAS est fait à partir d'une mallette \textit{NANUK 904}, qui a
été modifiée pour accommoder les autres composantes de la SAS. Les avantages de
cette mallette sont qu'elle s'ouvre tel une boîte à pizza et comprend 2
compartiments (haut et bas). De plus, cette mallette est de petite taille, est
légerte et est radioperméable (i.e. la coquille laisse passer les ondes radio).

\begin{figure}[H]
	\center
	\includegraphics[width=0.5\textwidth]{fig/boitier_sas.jpg}
	\caption{Une mallette NANUK 904}
	\label{f:boitier_sas}
\end{figure}

La figure \ref{f:boitier_sas} montre la mallette avec laquelle la SAS est
contruite.

\subsubsection{Ordinateur embarqué}

L'ordinateur embarqué est un Raspberry Pi 3B, utilisant Rasbian comme
distribution. Pour fonctionner, la SAS a besoin d'un ordinateur assez puissant
pour afficher l'interface graphique sans diminuer les performances de la SAS.
L'ordinateur doit également comprendre au minimum:

\begin{itemize}
	\item Un support de sauvegarde
	\item Un support WiFi intégré
	\item Un transmetteur Bluetooth
	\item Plusieurs prises USB type A
	\item Une sortie vidéo
\end{itemize}

Le Raspberry Pi 3B rempli très bien tous ces critères. De plus, beaucoup de
support est disponible en ligne pour les Raspberry Pi et les distributions
Linux.

\begin{figure}[H]
	\center
	\includegraphics[width=0.7\textwidth]{fig/rpi3b.jpg}
	\caption{Un Raspberry Pi 3B}
	\label{f:rpi3b}
\end{figure}

\subsubsection{Écran}

L'affichage de l'information se fait sur un écran tactile de 7", encastré dans
le compartiment supérieur du boîtier. Celui-ci est connecté par un câble HDMI au
Raspberry Pi.

\subsubsection{Clavier}

Le clavier de la SAS est entièrement fait par le GAUL. Il consiste en une
matrice de boutons et diodes, connectée à un microcontrôleur Teensy 2.0. Le
code a été généré par les outils en ligne suivants: le
\href{http://kbfirmware.com/}{Keyboard Firmware Builder} et le
\href{http://www.keyboard-layout-editor.com/}{Keyboard Layout Editor}. Le
firmware est complètement compatible avec Windows et Rasbian.

\begin{figure}[H]
	\center
	\includegraphics[width=0.9\textwidth]{fig/keyboard.jpg}
	\caption{Vue du dessous du clavier, visualisé dans Altium}
	\label{f:keyboard}
\end{figure}

\subsubsection{Alimentation}

Le circuit d'alimentation est composé de 6 piles LiPo et fournis de l'énergie
pour plus de 2~heures. Afin de pouvoir alterner entre une source d'alimentation
externe et la batterie, le circuit est équipé d'un circuit intégré qui fait ce
travail. De plus, chaque cellule LiPo est accompagnée d'un circuit intégré de
protection AP9101C.

\begin{figure}[H]
	\center
	\includegraphics[width=0.9\textwidth]{fig/alim_sas.jpg}
	\caption{Le PCB d'alimentation de la SAS, visualisé dans Altium}
	\label{f:alim_sas}
\end{figure}

\subsubsection{Antenne}



\subsubsection{GPS}



\subsubsection{Boussole}

La boussole (ou \textit{compass}) de la SAS est utilisée pour s'orienté lors de
la recherche de la fusée. Le capteur utilisé est un \textit{HMC5883L} et est
interfacé par I2C. Sa datasheet peut-être consultée
\href{https://cdn-shop.adafruit.com/datasheets/HMC5883L_3-Axis_Digital_Compass_IC.pdf}{ici}.

\subsubsection{Trackpad}

Le trackpad de la SAS est un \textit{Perixx PERIPAD-501B}, placé à droite du
clavier. Il est connecté par USB au Raspberry Pi. Les tests préliminaires ont
montré qu'il fonctionnait parfaitement sur Windows comme sur Raspbian. Par
contre, le projet fut annulé puisque le trackpad a été rendu inutile en le
démontant pour réduire l'espace qu'il prenait.

\begin{figure}[H]
	\center
	\includegraphics[width=0.5\textwidth]{fig/trackpad.jpg}
	\caption{Le trackpad de la SAS}
	\label{f:trackpad}
\end{figure}

\subsection{Station au sol - Logicielle}

Le logiciel de réception des données sur la station au sol est écrit en
\textit{Python 3}. Celui-ci est portable et donc peut s'exécuter sur un
ordinateur portable autant que sur le Raspberry Pi de la station au sol
matérielle.

\section{Rétrospective}

\subsection{Points forts}

L’assemblage physique du boitier et de ses composantes était remarquable.
En effet, le montage était bien fait, permettait de l’espace où c’était
nécessaire et était bien organisé où il y avait beaucoup de composantes.
\par
Le circuit d’alimentation était 100~\% fonctionnel dès la première itération.
L’alimentation était robuste et avait une longue durée de minimum deux heures
et demi.
\par
Les connecteurs à l’intérieur et à l’extérieur permettaient une accessibilité
et une versatilité accrues.
\par
Le système d’exploitation, soit Raspbian Linux, était bien et donnait un
environnement facile pour développer du code.
\par
Les capteurs dans la SAS étaient nombreux et font en sorte que l’on pouvait
acquérir beaucoup de données en parallèle à celles acquises par l’avionique.
\par
Le design électronique en trois circuits imprimés était merveilleux pour le
diagnostic électronique. Effectivement, la segmentation des systèmes aidait à
isoler les problèmes.
\par
Le format petit de la SAS était apprécié pour sa versatilité et sa
convivialité. Il n’était pas du tout problématique de la transporter.
\par
Le clavier personnalisé de la SAS garantissait une apparence impressionnante de
par sa taille et ses boutons.
\par
La conservation de la caractéristique étanche du boitier était nécessaire et
grandement appréciée. Dans le désert avec énormément de fine poussière,
l’électronique de la SAS se voit protégée. Les conditions d’utilisation de
cette dernière étaient donc plus permissives.
\par
La SAS pourrait jouer le rôle de la manette des relais. Il suffirait d'avoir une
sortie UART et un câble pour se connecter à la fusée.

\subsection{Points à améliorer}

Il manquait un pavé tactile afin de mieux contrôler la SAS. Par contre,
l’espace était prévu sur le panneau principal (HIDP).
\par
Les interrupteurs sur le HIDP étaient mal placés. Ils devraient se retrouvés
vers l’arrière et non vers l’avant quand on ouvre le boitier.
\par
Les batteries devraient avoir un mécanisme afin de les sécuriser. La solution
peut être aussi simple que d’installer des tie-wraps, mais il serait bien
d’avoir des clips de rétention avec glissières.
\par
Dans le désert, il serait bien d’avoir un mini panneau solaire (2-3 watts) sur
le dessus de la boite afin de maximiser la durée de vie des batteries et
d’augmenter l’autonomie de la SAS.
\par
Le chargeur de batteries doit être revu, puisqu’il permet seulement la recharge
d’une batterie. La SAS est tout de même utilisable.
\par
Le commutateur de source d’énergie doit être revu, puisqu’il permet seulement
le transfert instantané de la batterie à l’utilité (mais pas l’inverse).
\par
Ajouter un système de ventilation actif. Le design 2018 le prévoyait, mais un
manque de temps à empêcher de l’installer. De toute manière, il doit être revu
afin d’avoir les ouvertures (entrée et sortie) dans le HIDP.
\par
Le code de la SAS doit être davantage développé afin d’incorporer toutes les
fonctionnalités dans un petit écran tactile.
\par
Le support de l’écran dans le compartiment du haut doit être refait, parce que
l’écran possédait un jeu et parce que le câblage.
\par
Un espace doit être prévu pour l’antenne du GPS. Celle-ci avait été oubliée et
était donc installée libre dans le boitier.
\par
Les connexions entrant et sortant de chaque circuit imprimé devraient toutes
être du même côté pour faciliter la manutention des étages de circuits imprimés.
\par
L’espace sur les circuits imprimés pourrait être davantage occupée en
distançant les composantes dessus. Ainsi, la soudure et l’assemblage seraient
faciles.
