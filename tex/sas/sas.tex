\chapter{Station au sol}
\label{chap:sas}

\section{Réalisation}



\section{Rétrospective}

\subsection{Points forts}



\subsection{Points à améliorer}


Rétroaction SAS 2018
Version préliminaire 2018-08-02

% ---------------------------------------------------------
%
% Points forts:
% L’assemblage physique du boitier et de ses composantes était remarquable. En effet, le montage était bien fait, permettait de l’espace où c’était nécessaire et était bien organisé où il y avait beaucoup de composantes.
% Le circuit d’alimentation était 100% fonctionnel dès la première itération. L’alimentation était robuste et avait une longue durée de minimum deux heures et demi.
% Les connecteurs à l’intérieur et à l’extérieur permettaient une accessibilité et une versatilité accrues.
% Le système d’exploitation, soit Raspbian Linux, était bien et donnait un environnement facile pour développer du code.
% Les capteurs dans la SAS étaient nombreux et font en sorte que l’on pouvait acquérir beaucoup de données en parallèle à celles acquises par l’avionique.
% Le design électronique en trois circuits imprimés était merveilleux pour le diagnostic électronique. Effectivement, la segmentation des systèmes aidait à isoler les problèmes.
% Le format petit de la SAS était apprécié pour sa versatilité et sa convivialité. Il n’était pas du tout problématique de la transporter.
% Le clavier personnalisé de la SAS garantissait une apparence impressionnante de par sa taille et ses boutons.
% La conservation de la caractéristique étanche du boitier était nécessaire et grandement appréciée. Dans le désert avec énormément de fine poussière, l’électronique de la SAS se voit protégée. Les conditions d’utilisation de cette dernière étaient donc plus permissives.
%
% Points faibles et pistes d’amélioration:
% Il manquait un pavé tactile afin de mieux contrôler la SAS. Par contre, l’espace était prévu sur le panneau principal (HIDP).
% Les interrupteurs sur le HIDP étaient mal placés. Ils devraient se retrouvés vers l’arrière et non vers l’avant quand on ouvre le boitier.
% Les batteries devraient avoir un mécanisme afin de les sécuriser. La solution peut être aussi simple que d’installer des tie-wraps, mais il serait bien d’avoir des clips de rétention avec glissières.
% Dans le désert, il serait bien d’avoir un mini panneau solaire (2-3 watts) sur le dessus de la boite afin de maximiser la durée de vie des batteries et d’augmenter l’autonomie de la SAS.
% Le chargeur de batteries doit être revu, puisqu’il permet seulement la recharge d’une batterie. La SAS est tout de même utilisable.
% Le commutateur de source d’énergie doit être revu, puisqu’il permet seulement le transfert instantané de la batterie à l’utilité (mais pas l’inverse).
% Ajouter un système de ventilation actif. Le design 2018 le prévoyait, mais un manque de temps à empêcher de l’installer. De toute manière, il doit être revu afin d’avoir les ouvertures (entrée et sortie) dans le HIDP.
% Le code de la SAS doit être davantage développé afin d’incorporer toutes les fonctionnalités dans un petit écran tactile.
% Le support de l’écran dans le compartiment du haut doit être refait, parce que l’écran possédait un jeu et parce que le câblage.
% Un espace doit être prévu pour l’antenne du GPS. Celle-ci avait été oubliée et était donc installée libre dans le boitier.
% Les connexions entrant et sortant de chaque circuit imprimé devraient toutes être du même côté pour faciliter la manutention des étages de circuits imprimés.
% L’espace sur les circuits imprimés pourrait être davantage occupée en distançant les composantes dessus. Ainsi, la soudure et l’assemblage seraient faciles.
%
% ---------------------------------------------------------
